\documentclass[../Paper.tex]{subfiles}
\begin{document}
\justifying
All model functions were coded in R\cite{R} with {C++} suplement.
The utilized packages were as follows:
\begin{enumerate}
  \item[$\bullet$]{adaptivetau\cite{johnson_2016}}
  \item[$\bullet$]{ggplot2\cite{Wickham_2009}}
  \item[$\bullet$]{data.table\cite{data.table}}
  \item[$\bullet$]{Rcpp\cite{Rcpp}}
  \item[$\bullet$]{Foreach\cite{foreach}}
  \item[$\bullet$]{microbenchmark\cite{microbenchmark}}
  \item[$\bullet$]{profvis\cite{profvis}}
\end{enumerate}
To increase modelling performance, the Foreach package parallelized the modelling
function to utilize all availible processor threads. The data.table package provided
the ``data.table'' R object that is capable of much faster data sorting and
subsetting methods than the standard packages. Rcpp served as a linker package
between R and {C++}, allowing for computationally intensive portions of the R code
to be offloaded to {C++} and called within the R environment.
Finally, the microbenchmark and profvis packages were essential for code
optimization and profiling.

In addition to the model functional code, this document was created in \LaTeX\
and compiled to PDF using the \XeLaTeX\  compiler.
\clearpage
\end{document}
