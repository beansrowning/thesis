\documentclass[../Paper.tex]{subfiles}

\begin{document}
  \justifying
  \begin{figure}[htbp]
    \centering
    \begin{tikzpicture}[node distance=1cm, auto, >=latex]
    \node [mystyle]              (S1) {$S_{1}$};
    \node [mystyle, right=of S1] (E1) {$E_{1}$};
    \node [mystyle, right=of E1] (I1) {$I_{1}$};
    \node [mystyle, right=of I1] (R1) {$R_{1}$};
    \node [mystyle, below=of S1] (S2) {$S_{2}$};
    \node [mystyle, below=of E1] (E2) {$E_{2}$};
    \node [mystyle, below=of I1] (I2) {$I_{2}$};
    \node [mystyle, below=of R1] (R2) {$R_{2}$};
    \coordinate[above=of S1] (in1);
    \coordinate[above=of R1] (in2);
    \coordinate[below=of S2] (out1);
    \coordinate[below=of E2] (out2);
    \coordinate[below=of I2] (out3);
    \coordinate[below=of R2] (out4);
    \path[->]
              % Young Compartment
              (S1)   edge node {$\beta I_{1}$}                   (E1)
                     edge node {$\theta$}                        (S2)
              (E1)   edge node {$\phi$}                          (I1)
                     edge node {$\theta$}                        (E2)
              (I1)   edge node {$\gamma$}                        (R1)
                     edge node {$\theta$}                        (I2)
              (R1)   edge node {$\theta$}                        (R2)
              % Old Compartment
              (S2)   edge node {$\beta I_{2}$}                   (E2)
              (E2)   edge node {$\phi$}                          (I2)
              (I2)   edge node {$\gamma$}                        (R2)
              % Births
              (in1)  edge node {$\alpha\cdot\left(1-\nu\right)$} (S1)
              (in2)  edge node {$\alpha\cdot\nu$}                (R1)
              (S2)   edge node {$\Omega$}                        (out1)
              % Deaths
              (E2)   edge node {$\Omega$}                        (out2)
              (I2)   edge node {$\Omega$}                        (out3)
              (R2)   edge node {$\Omega$}                        (out4);
    \end{tikzpicture}
    \caption{Age stratified SEIR Model with vital dyanmics}
    \label{fig:M1}
  \end{figure}
  Where:

   $\beta_{n} = $ Rate of contact between infectious and susceptible persons \\
   $\phi = $ Rate of onset of infectiousness subsequent to being infected \\
   $\gamma = $ Rate of recovery from measles from infectious period \\
   $\alpha = $ Crude birth rate \\
   $\nu = $ Effective vaccination rate at birth \\
   $\Omega = $ Crude death rate \\
   $\theta = $ Rate of aging from young to old compartments

  To explore the infection dynamics of measles, a SEIR model was selected with two
  age compartments, subdividing younger and older populations. This was chosen for a few
  reasons. Firstly, it allows for the modelling of heterogeneous mixing within
  these age groups. Additionally, it allows for the model to more closely
  align with the age strata reported in seroprevalence data, thereby reducing the
  number of assumptions which would need to be made to fit incoming data.
  Finally, the selection of two compartments over some higher dimension limits
  the mathematical complexity of both modeling the data as well as mapping the
  parameter space while also allowing for some variation in age demographics.

  \clearpage

\end{document}
