\documentclass[../Paper.tex]{subfiles}
\begin{document}
  \justifying
  To determine the smallest inputs into the model which would result in
  an epidemic, I analyzed the hyperparameter space of two model parameters.
  I employed whole grid search to avoid the posibility of missing the global optimum,
  which approximation methods such as random search may do without appropriate
  parameter weighting.

  Let the two parameter inputs be defined as:
  \begin{equation}
    x \in \left\{1, 2, \cdots, n \right\}
  \end{equation}
  and,
  \begin{equation}
    y \in \left\{1, 2, \cdots, n \right\}
  \end{equation}
  where:\\
  $x =$ number of new cases of measles added to the model, and\\
  $y =$ number of ``insertion events'' which occur during one model simulation.

  The cost function, $f(x, y)$, provides scalar output of the largest outbreak length
  when evaluating the model with $x$ and $y$ as inputs.

  The three dimensional surface of the hyperparameter space is found by evaluating $f(x,y)$
  with all reasonable combinations of $x$ and $y$. The three values can then be
  expressed as cartesian coordinates and plotted. Importantly, this will help to reveal
  local maxima, minimizing for the input values.

  % Subfile for plot is not necessary, but it makes the file look better
  \subfile{sections/paramspacegraph}
  \clearpage
\end{document}
