\documentclass[../Paper.tex]{subfiles}
\begin{document}
\justifying
There were three nations chosen for analysis using the model mentioned above:
Sweden, Malta, and Latvia. All three have been declared measles-free by the WHO
and see few annual cases. Despite this, the populations of these three countries
have entirely different seroprevalence values with Latvia having the highest
proportion of seronegative individuals and Sweden having the lowest.
The table below highlights this variation along with nationally-reported MCV1 and MCV2 rates.
\begin{table}[htbp]
\centering
\label{my-label}
\begin{tabular}{@{}lllllllllllllllll@{}}
\multicolumn{2}{l}{\multirow{2}{*}{}} & \multicolumn{7}{c}{MCV1} &  & \multicolumn{7}{c}{MCV2} \\ \cmidrule(lr){3-9} \cmidrule(l){11-17}
\multicolumn{2}{l}{} & 2016 & 2015 & 2014 & 2013 & 2012 & 2011 & 2010 &  & 2016 & 2015 & 2014 & 2013 & 2012 & 2011 & 2010 \\ \cmidrule(r){1-17}
\multicolumn{2}{l}{Sweden} & 97\% & 98\% & 97\% & 97\% & 97\% & 96\% & 97\% &  & 95\% & 95\% & 95\% & 95\% & 95\% & 95\% & 94\% \\
\multicolumn{2}{l}{Malta} & 93\% & 89\% & 98\% & 99\% & 93\% & 84\% & 73\% &  & 86\% & 91\% & 94\% & 88\% & 91\% & 85\% & 97\% \\
\multicolumn{2}{l}{Latvia} & 93\% & 96\% & 95\% & 96\% & 90\% & 92\% & 95\% &  & 89\% & 92\% & 89\% & 92\% & 92\% & 92\% & 93\% \\ \bottomrule
\end{tabular}
\caption{MCV1 and MCV2 coverage rates as reported to the WHO \cite{world_health_organization_table}}
\end{table}


Population demographic data for the project was acquired from the United Nations
Department of Economic and Social Affairs, Population Division (UN DESA). Measles
seroprevalence data was made available by findings from ESEN2 presented to the World
Health Organization.

As previously mentioned, the model will have two age compartments in an effort to
reduce the number of dimensions. However, the WHO report provides seroprevalence data
in several age categories, and the appropriate proportion of measles seronegativity
in the younger age compartment must be extrapolated. It stands to mention that
many of the data are quite dated.
Several countries, such as Sweden, have not submitted updated data in over 20 years,
therefore some liberty was taken in best estimating these population values.
Where possible, sensitivity analyses of the parameters were conducted to
quantify the impact on model results.



\clearpage
\end{document}
