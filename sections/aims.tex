\documentclass[../Paper.tex]{subfiles}
\begin{document}
  \justifying
  The primary aim is to explore the susceptibility of measles-eliminated
  countries in Europe to experience an epidemic following new case importation.
  There are two ideas behind this aim:

  Firstly, I reason that there is some interplay between population size,
  the rate of new case introduction, and effective vaccination rate which
  affects it's susceptibility to an epidemic;

  Secondly, I reason that there exists some period of time, after which ``measles eliminated''
  countries would again be susceptible to large scale outbreaks, dependent on
  the interplay descibed in the former premise.



  To accomplish this, I first created a stochastic transmission model of measles
  in the country. Using population data from the United Nations and seroprevalence
  data from the World Health Organization, I extrapolated initial population
  values for the compartmental model.

  Beyond this, I created a modelling framework in the R programming language
  in order to handle the insertion of infected individuals into the model and
  determine the impact after running several iterations. As the
  \clearpage
\end{document}
